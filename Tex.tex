\documentclass[
	parskip=half,10pt,
	numbers= noenddot, % enddot -> Ebenen mit Punkt abschließen -> 1.1., noenddot -> ohne Punkt
	toc=flat, % TOC in Tabellenform (bei langen Überschrtiften verwenden)
	oneside,
	twocolumn,
	]{scrartcl}



\usepackage[T1]{fontenc} % verwende Type 1 - Zeichensatz
\usepackage{libertine}
\usepackage[scaled=0.78]{beramono} %Schreibmaschinenschrift
\usepackage{microtype}
\usepackage[utf8]{inputenc}
\usepackage[english]{babel} % internationale Sprachunterstützung



\usepackage{amsmath}
\usepackage{amssymb}
\usepackage{amsthm}
\usepackage{tabularx}
\usepackage{booktabs}
\usepackage{longtable}
\usepackage{rotating} % Rotationen, Reflexionen, ...
\usepackage{multido} % Wiederholungen
\usepackage{wrapfig}
\usepackage{todonotes}
\usepackage{siunitx} %\si units
\usepackage{units}
\usepackage{icomma} %keine Leerzeichen nach Komma im mathmode
\usepackage[numbers,sort]{natbib}
\usepackage{babelbib} %deutsche bibliographie
\usepackage{multirow}
\usepackage{rotating}
\usepackage{url}

\usepackage{tikz}
\usepackage{float}
\usepackage{pgfplots}
%\pgfplotsset{compat=1.8}
\usepackage{caption}
\usepackage{graphicx}
\usepackage{subcaption} %für subfigures
\captionsetup{labelfont={bf,sf},format = plain, textfont=sf}
%\usepackage{asymptote}
\usepackage{ragged2e}
\usepackage[bottom]{footmisc}
\usepackage{csquotes} %Anführungszeichen
\usepackage[ngerman]{varioref} % Zum komfortablen Verlinken



\usepackage{geometry}
\geometry{a4paper,lmargin=2.5cm, rmargin=2.5cm, tmargin=2.5cm, bmargin=3cm, marginparwidth=3cm, marginparsep=1em}


\usepackage{fancyhdr}
\pagestyle{fancy}
\renewcommand\footrulewidth{0.5pt}
\fancyhf{}
\lhead{\leftmark}

\fancyfoot{}
\rfoot{\thepage}
\lfoot{Till Kolster \& Lukas Schmidt}


\usepackage{layout}

\usepackage[%draft
linkbordercolor=blue,
colorlinks,
linkcolor=blue,
linktocpage,
linktoc=all]{hyperref} % IMMER AM ENDE

%\setkomafont{subparagraph}{\mdseries\itshape}
\setcounter{secnumdepth}{3}%Bis zu welcher Ebene nummeriert werden soll. 
\setcounter{tocdepth}{2}%Bis zu welcher Tiefe ins TOC soll.

%%%%%EIGENE DEFINITIONEN%%%%%

\newcolumntype{Y}{>{\RaggedRight\hspace{0pt}} X }
\newcommand\Grad{$^\circ$}
\newcommand\HAND{\marginnote{\Large\vreflectbox{\ding{43}}}\xspace}%\newcommand\Name{Befehlsdifinition}
\newcommand\MPAR[1]{
\marginnote[\RaggedLeft#1]{\RaggedRight#1}}
% \hspace{0pt} entspricht dem ersten (nicht sichtbaren) Wort
\newcolumntype{P}[1]{>{\RaggedRight\hspace{0pt}}p{#1}}
\newcolumntype{R}{>{\tiny}r}

\pgfmathdeclarefunction{gauss}{4}{%
  \pgfmathparse{#1*exp(-((x-#2)^2)/(2*#3^2))+#4}%
}


\title {Rayleigh-Scattering}
\author {Till Kolster \thanks{Freie Universität Berlin} \and Lukas Schmidt \thanks{Freie Universität Berlin}}


\begin{document}

\begin{titlepage}

\vspace*{-2cm}

\vspace{6cm}
\begin{center}
\huge \bfseries
Fortgeschrittenen-Praktikum -- Rayleigh-Scattering

\vspace{0.5cm}
\large \bfseries
03.12.2014

\vspace{1.5cm}

\large\normalfont von

\bigskip
\textbf{Till Kolster \& Lukas Schmidt}

\bigskip
Tutor: Dr. Andrey Pivtsov

\vspace{3cm}

\parbox{0.8\linewidth}{%
\textit{This experiment is done within the scope of the advanced lab course for Bachelor Students at Freie Universität Berlin.
It should give an experimental introduction to Rayleigh scattering processes, the Scattering-Ring-Down Spectroscopy and should
give a better understanding of Rayleigh scattering phenomen in nature.
}}


\end{center}
\end{titlepage}


\section{Physikalische Grundlagen}
\subsection{Rayleigh-Scattering}
Rayleigh Scattering is
\subsection{general concept of cross-section parameters}
Cross section parameters, often described by the letter $\sigma$ quantify the interaction between particles or waves with another particle, called target.
To describe different processes, different parameters have been defined, but the principle that lies behind is the same.
A higher cross section parameter refers to a higher incident of the specified interaction or analogiously to a higher probability that this interaction happens in a certain time. 
Its basic definition is $\sigma=w\cdot \nicefrac{F}{N_T}$ and its unity is $[\sigma]=\si{\meter}^2=10^28$barn where F is the surface of the target and N the number of particles that are included in this area.
\subsection{Cavity Ring-Down Spectroscopy CRDS}
Cavity Ring-Down spectroscopy is a method to observe interactions of photons with a medium of gas. It consists of a cavity equipped with very reflective mirrors to be able to trap light in the cavity. A small amount of the light can leave the cavity and will be measured with the help of a photomultiplier. As some of the photons will leak out or be absorbed by the not perfect mirrors the measured signal can be described by an exponential decay with the decay constant $\tau$:
\begin{equation}
I(t)=e^{-\nicefrac{t}{\tau}}
\end{equation}



\newpage
\bibliographystyle{unsrtnat}
\bibliography{raybib}

\end{document}





